\documentclass[a4paper]{scrartcl}

% Für mathematische Symbole
\usepackage{mathtools}
\usepackage{amsfonts}

\usepackage{import}

\usepackage{polyglossia}
\setdefaultlanguage[ 
	spelling = new, 
	babelshorthands = true ]
{german}

% Für Pseudo-Code
\usepackage[ruled,vlined]{algorithm2e}

%Hyperlinks
\usepackage{hyperref}
\hypersetup{
	colorlinks   = true, %Colours links instead of ugly boxes
	urlcolor     = blue, %Colour for external hyperlinks
	linkcolor    = blue, %Colour of internal links
	citecolor   = red %Colour of citations
}

% Glossar
% nomain - wenn importiert
% automake - für autocompile
%\usepackage[xindy, automake]{glossaries}
%\subimport{../../../../KIT/LaTex/Glossary/}{MathematikGlossary}
%\subimport{../../../../KIT/LaTex/Glossary/}{TheoretischeInformatikGlossary}
%\subimport{../../../../KIT/LaTex/Glossary/}{TechnischeInformatikGlossary}

%Aritmethikoperationen
%\usepackage{basicarith}

%Für Farben
\usepackage[dvipsnames]{xcolor}
\definecolor{TextMarkerGelb}{RGB}{255, 255, 120}
\usepackage{soul}
\sethlcolor{TextMarkerGelb}

% Fonts
\usepackage{fontspec}
\usepackage{unicode-math}

\setmainfont{Roboto}
\setsansfont{Roboto}
\setmonofont{Roboto}

% Math-Font
\setmathfont{Roboto} % for math symbols, can be any other OpenType math font

% Aside Kommentar
\usepackage{marginnote}

% Package für bessere Liste
\usepackage{scrextend}

% Mehrere Zeilen in einer Table für einen Eintrag nutzen
\usepackage{multirow}

% Durchstreichen
\usepackage[normalem]{ulem}

% Coole Listen [Seperator-zeichen]
\usepackage[ampersand]{easylist}
%\ListProperties(Style**=$\diamond$)

%Ermöglicht [H]
\usepackage{float}

%\DeclareMathSizes{10}{18}{12}{8}   % For size 10 text
%\DeclareMathSizes{11}{19}{13}{9}   % For size 11 text
%\DeclareMathSizes{12}{20}{14}{10}  % For size 12 text

% Paragraph-Einstellung
\usepackage[explicit]{titlesec}
\setcounter{secnumdepth}{4}
\titleformat{\paragraph}[block]
{\normalfont\normalsize\bfseries}{}{0pt}{\uline{#1}}
\titleformat{name=\paragraph,numberless}[block]
{\normalfont\normalsize\bfseries}{}{0pt}{\uline{#1.}}

\usepackage{fontspec}
\usepackage{unicode-math}
\setmainfont{Roboto}
\setsansfont{Roboto}
\setmonofont{Roboto}
\setmathfont{Asana Math} % for math symbols, can be any other OpenType math font
\setmathfont[range=\mathup]  {Roboto}
\setmathfont[range=\mathbfup]{Roboto}
\setmathfont[range=\mathbfit]{Roboto}
\setmathfont[range=\mathit]  {Roboto}

% \emph{} - ändern
\DeclareTextFontCommand{\emph}{\bfseries}

% Neuer Befehl für not <->
\newcommand{\nleftright}{\mathrel{\ooalign{$\Leftrightarrow$\cr\hidewidth$/$\hidewidth}}}

%\makeglossaries

%opening
\title{Mitschrift Digitaltechnik und Entwurfsverfahren}

\author{Paul Züchner}


\begin{document}
	
	\maketitle
	\tableofcontents
	\newpage
	
	\section{Anforderungen höherer Programmiersprachen}
		\subsection{Programmiersprache C}
		\begin{labeling}{Maschinensprache:}
			\item[Maschinensprache:]Anweisung die für den Prozessor ohne weiteres verständlich ist
			\item[Assemblersprache:]  Symbolische Repräsentation der Maschinensprache, die für den Menschen verständlich und anschaulich ist z.B. add \$s2, \$s1, \$s0 \# \$s2:= \$s1 + \$s0
		\end{labeling}
	
		C ist sehr effizent.\\
		Hat die Datentypen:
		\begin{easylist}[itemize]
			& char 
				&&& meist 1 Byte
			& int
				&& short int
				&& long int
				&& long long int
					&&& kombinierbar mit un/signed
					&&& meist 2 oder 4 Byte 
			& float
				&& Gleitkommazeilen
					&&& meist 4 Byte
			& double
				&& Gleitkommazeilen mit doppelter Genauigkeit
					&&& meist 8 Byte
		\end{easylist}
		
		Es gilt in C : \[ sizeof(char) \leq sizeof(short int) \leq sizeof(int) \leq sizeof(long int) \leq sizeof(long long int) \]
	
		Parameter werden als Werte an die Funktionen übergeben, daher kann eine Funktion diese nicht ändern\\
		Zeiger/Pointer deren Zielobjekt kann geändert werden\\
		
	\section{Grundlegendes Rechnermodell}
		\subsection{Zentraleinheit (central processing unit, CPU)}
		Besteht aus Stuerwerk und Rechenwerk\\
		Verarbeitet Daten Gemäß eines Programms\\
		
			\subsubsection{Steuerwerk (Leitwerk, Control unit, CU)}
			Holt die Befehle eines Programms aus dem Speicher \\
			Dekodiert sie\\
			Steuert ihre Ausführung in der verlangten Reihenfolge durch	Steuer- und Status-Signale \\
			\\
			Befehlsregister\\
			Dekoder\\
			Taktgenerator\\
			
			\subsubsection{Rechenwerk (Operationswerk, Ausführungseinheit, ALU)}
			Führt die Rechenorperationen (z.B. arithmetisch/logische Operationen aus)\\
			Auszuführende Operationen werden durch die 	Steuersignale bestimmt \\
			Liefert seinerseits Statussignale an das Steuerwerk zurück\\
			\\
			\paragraph{Statusregister}
			Flags:\\
			-- Übertragsbit (Carry Flag CF)\\
			-- Hilfsübertragsbit (Auxiliary Carry AF)\\
			-- Nullbit (Zero Flag ZF)\\
			-- Vorzeichenbit (Sign Flag SF)\\
			-- Überlaufbit (Overflow Flag OF)\\
			-- Even Flag EF\\
			-- Paritätsbit (Parity Flag PF)\\
			
		\subsection{Speicherwerk}
		Jede Speicher ist \emph{eindeutig} durch ihre Adresse identifizierbar\\
		Dort werden Programme und Daten abgelegt (von-Neumann-Konzept)\\
		Alternativ: Harvard-Architektur	mit getrenntem Programm- und Datenspeicher\\
		
		\subsection{Verbindungsnetzwerk}
		Verbindet die Komponenten für den Austausch von Daten\\
		Bus \( \iff  \) eine Leitung auf denen Daten transportiert werden\\
		-- Adressleitungen: Adressinformationen \\
		-- Datenleitungen: Daten und Befehle zum Prozessor \\
		--Steuerleitungen: Steuer- oder Statusbefehle \\
		
		\subsection{Ein-Ausgabewerk}
		Drucker, Bildschirm...\\
		
		\subsection{Maschinenbefehlszyklus}
		Holphase \( \rightarrow \) Dekodierphase  \( \rightarrow \) Ausführungsphase\\
		
		\subsection{Mima}
		Speicherzugriffe brauchen 3 Taktzyklen\\
		\\
		24 Bit Befehle\\
		OpCode 4 Bits, Addresse oder Konstante 20 Bit\\
		\begin{table}[H]
			\begin{tabular}{|l|c|c|}
				\hline
				Name & Ansteuerrungsbusse & Mit Datenbus verbunden \\
				\hline
				Akku  & A\( _r \) \& A\(_w\) \& N & 24 Bit, bidirektional\\
				Eins  & E & 24 unidirektional\\
				Speicheradressregister  & S & 20, uni\\
				SpeicherdecodeRegister & D\( _r \) \& D\(_w\) & 20, uni\\
				Instructenadressregister & P\( _r \) \& P\(_w\) & 20, bi\\
				Instructenregister & I\( _r \) \& I\(_w\) & 24, bi \\
				\hline
			\end{tabular}
		\end{table}
		
		\subsection{Abkürzungen}
		\emph{HR:} Hilfsregister\\
		\emph{AC:} Accumulator\\
		\emph{BR:} Befehlsregister\\
		\emph{AP:} Adress Puffer\\
		\emph{PC:} Program	Counter\\
		\emph{ABT:} Adressbus Puffer\\
		\emph{DBP:} Datenbus Puffer\\
		\\
		\emph{IAR:} InstruktionsAdressRegister \\
		\emph{SAR:} SpeicherAdressRegister \\
		\emph{SDR:} SpeicherDatenRegister\\
		
	\section{Instruction Set Architektur }
		\subsection{Datentypen \& -formate}
		halfword\\
		word\\
		longword\\
		
		\subsection{Adressierungsarten}
		Explizite oder Implizierte Adressierung\\
		Überdeckte Adressierung: Wenn Quelle und Ziel (Register) sich überdecken\\
		\\
		Explizite Adressierung: Der Operand steht in einem Register\\
		Direkte Adressierung: immidiate \\
		Basis-Adressierung: Adr(Opr) = Konst. + Registerinhalt\\
		Befehlszähler-relative Adressierung: Adr = Befehlszähler + Kostante aus Befehl\\
		
		
		\subsection{Befehlsformat \& -satz}
		\[\text{CPU\_Zeit} = IC \times CPI \times \text{Taktdauer} \]
		\[\text{CPU\_Zeit} = \frac{IC \times CPI}{\text{Taktfrequenz}} \]
		IC: Anzahl der ausgeführten Befehle (Instruction Count)\\
		CPI: Anzahl der Taktzyklen pro Befehl (Cycle per Instruction)\\
		
		
		\subsection{RISC}
		Ziel: Reduzierung der Zyklen je Befehl (CPI)\\ 
		
		\subsection{CISC}
		Um Assembler Programmierer zu unterstützen komplexe Befehle\\
		Ziel: Reduzierung der ausgeführten Befehle (Instruction Count)\\ 
		Festverdrahtete Logik\\
		
		\subsection{Fallstudie-Mips}
		
	\section{Einführung in die Assemblerprogrammierung}
		\subsection{Mips-Assembler (Microprocessor without Interlocked Pipeline Stages)}
			\subsubsection{Datentypen}
			.byte\\
			.half\\
			.word (entspricht einer Zeile im Speicher 32bit)\\ 
			.float\\
			.double\\
			.asciiz (String)\\
			
			
			\subsubsection{Assemblerdirektiven} 
			z.B. .align .ascii\\
			
			.space n : allokiert n Bytes\\
			.globl xy : xy ist ein globales Symbol\\
			
			
			\subsubsection{Marken (Labels)} 
			
			\subsubsection{Maschinenbefehlen (und Pseudobefehlen, Makros)} 
			
			\subsubsection{Kommentaren}
			\#Kommentar\\
			Keine Mehrzeilige-Kommentare\\
			
			\subsubsection{Systemaufrufe (syscall)}
			\begin{table}[H]
				\begin{tabular}{|l|c|l|}
					\hline
					Dienst & Nummer & Eingabe \\
					\hline
					print\_int & 1 & Integer in \$a0\\
					print\_float & 2 & Float in \$f12\\
					print\_double & 3 & Double in \$f12 und \$f13\\
					print\_string & 4 &Adresse des Strings in \$a0 \\
					read\_int & 5 & Integer in \$v0\\
					read\_float & 6 & Float in \$f0\\
					read\_double & 7 & Double in \$f0 und \$f1\\
					read\_string & 8 &Adresse der Zeichenkette in \$a0 und  maximale Länge in \$a1 übergeben.* \\
					sbrk & 9 &	Byte-Anzahl in \$a0.  Anfangsadresse in \$v0\\
					exit & 10&\\
					\hline
				\end{tabular}
			\end{table}
			*Bei weniger eingegebenen Zeichen wird die Zeichenkette zusätzlich mit "\\n" abgeschlossen \\
		
		
	\section{Befehlsabarbeitung im grundlegenden Rechnermodells}
		\subsection{Logische Phasen des Maschinenbefehlszyklus}
		\subsection{Pipelininig}
		
	\section{Prozessorarchitektur}
		\subsection{Steuerwerk}
		\subsection{Rechenwerk}
		
	\section{Speicher}
		\subsection{Speicherkomponenten}
		\subsection{Systemkomponenten}
		\subsection{Halbleiterspeicher}
		\subsection{Cache}
		
	\section{Betriebssystemunterstützung}
		\subsection{Virtuelle Speicherverwaltung}
		\subsection{Unterbrechungsbehandlung}
		
	\section{Ein-Ausgabe}
		\subsection{Schnittstellenbausteine}
		\subsection{DMA}
		\subsection{Interrupt-Controller}
		
	\section{Bussysteme}
		\subsection{Grundlegende Eigenschaften}
		\subsection{Zuteilung \& Protokolle}
		
	
	
	
	
	
	
	
	
	
\end{document}